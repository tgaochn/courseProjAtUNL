\documentclass[a4paper]{article}

\usepackage[english]{babel}
\usepackage[utf8]{inputenc}
\usepackage{amsmath}
\usepackage{graphicx}
\usepackage{minted}
\usepackage{algorithm}
\usepackage[noend]{algpseudocode}
\usepackage[colorinlistoftodos]{todonotes}
\usepackage{multirow}
\usepackage{parskip}

% empty set package
\usepackage{amssymb}

\title{Intro to AI - Take-home Pretest}
\author{Tian Gao}
\begin{document}
\maketitle


% Question 1
1. (a)\\
Assume the claim $4_n=O(2^n)$ is correct. \\
$\therefore$ exists a $c>0$ and $n_0$, s.t. $4^n = 2^{2n} \leqslant c  2^n, \forall n>n_0$\\
$\therefore log_2(4^{n}) = log_2(2^{2n})=2n \leqslant log_2(c  2^n) = log_2c + log_2(2^{n}) = log_2c + n$ \\
$\therefore n \leqslant log_2c$ which is incorrect since $log_2c$ is a fixed number.\\
$\therefore$ the claim does not hold and $4_n!=O(2^n)$.\\

1. (b)\\
Assume the claim $2_n=O(4^n)$ is correct. \\
$\therefore$ exists a $c_1>0, c_2>0$ and $n_0$, s.t. $c_1 4^n \leqslant 2^n \leqslant c_2  4^n , \forall n>n_0$\\
$\therefore log_2(c_1  4^n) = log_2c_1 + log_2(2^{2n}) = log_2c_1 + 2n \leqslant log_2(2^{n}) = n$ \\
$\therefore n \leqslant -log_2c_1$ which is incorrect since $-log_2c_1$ is a fixed number.\\
$\therefore$ the claim does not hold and $4_n!=O(2^n)$.\\

% Question 2
2. \\
$\forall \alpha, \beta$, $\because \alpha > \beta > 0$\\
$\therefore \exists c=\frac{\beta}{2\alpha}>0$ s.t. $\alpha > \beta > c\alpha=\frac{\beta}{2}$\\
$\therefore \alpha^n > \beta^n > (c\alpha)^n$ if $n > 1$\\
$\therefore \exists c_1=1>0, c_2=c^n>0, n_0=1>0$ s.t. $c_1\alpha > \beta > c_2\alpha^n>0$\\
$\therefore \beta^n \in O(\alpha^n)$

% Question 3
3. \\
Let x' is a element in S and we need to prove P(x') holds if Basic Step and Induction Step hold.\\
$\because $ S is a well-ordered set.\\
$\therefore$ elements in S are ordered.\\
Let x' be in the $n_{th}$ position and the order of elements in S is $x_0 < x_1 < \dotsb < x_{n-1} < x' < x_{n+1} < \dotsb$\\
$\because P(x_0)$ is true\\
$\therefore P(x_1)$ holds, so does $P(x_2), \dotsb P(x_{n-1}), P(x'), P(x_{n+1}), \dotsb$(according to Induction Step)\\
As a result, $\forall x \in S$, P(x) holds.\\

% Question 4
4. (a)\\
(Please find the Algorithm on the next page)
  \begin{algorithm}[H]
  \caption{ \Large{\textbf{Two-classes partition algorithm}}}  
  \label{alg:Framwork}  
  \begin{algorithmic}[1]  
    \Require an undirected graph G(V,E);
    \Ensure whether the graph can be divided into two class, node set of class1, node set of class2
    \Function {GraphPartition}{V, E}
    
    \State $node_{visited} \gets \varnothing$;
    \State $node_{avail} \gets V$;
    \If {$node_{avail} = \varnothing$}
    	\State \Return{true, $node_{class1}$, $node_{class2}$};
    \EndIf
    
    \State $u_0 \gets$ a random node in $node_{avail}$;
    \State $node_{class1} \gets \{u_0\}$;
    \State $node_{adj} \gets$ \{$v|u_0,v \in V, (u_0, v) \in E$\};
    
    \While {$node_{adj} != \varnothing$}
    	\State $u_0 \gets$ a random node in $node_{avail}$;
        \State $node_{class1}$ add $u_0$;
    \EndWhile
    
    
    \State $node_{visited} \gets node_{class1}$;
    \State $node_{avail} \gets V-node_{visited}$;
    \If {$node_{avail} = \varnothing$}
    	\State \Return{false, \{\}, \{\}};
    \EndIf    
    \State $u_0 \gets$ a random node in $node_{avail}$;
    \State $node_{class2} \gets \{u_0\}$;
    \State $node_{adj} \gets$ \{$v|u_0,v \in V, (u_0, v) \in E$\};
    
    \While {$node_{adj} != \varnothing$}
    	\State $u_0 \gets$ a random node in $node_{avail}$;
        \State $node_{class2}$ add $u_0$;
    \EndWhile
    
    \State $node_{visited}$ add $node_{class1}$;
    \State $node_{avail} \gets V-node_{visited}$;
    
    \If {$node_{avail} = \varnothing$}
    	\State \Return{true, $node_{class1}$, $node_{class2}$};
    \Else \;
    	\State \Return {false, \{\}, \{\}};
    \EndIf 
   
    \EndFunction
  \end{algorithmic}  
\end{algorithm}  

4. (b)\\
If the graph has n nodes and e edges, the running time of my algorithm is $O(ne)$.

\end{document}

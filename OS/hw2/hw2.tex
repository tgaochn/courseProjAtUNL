\documentclass[a4paper]{article}

% necessary package
\usepackage[english]{babel}
\usepackage[utf8]{inputenc}
\usepackage[colorinlistoftodos]{todonotes}
\usepackage{parskip} % spacing between paragraphs controller
\usepackage{amsmath} % equation alignment and reference
\usepackage{enumerate} % list items
\usepackage{graphicx} % figure alignment and reference

%% algorithm pseudocode related
% \usepackage{algorithmicx}
% \usepackage[noend]{algpseudocode}
% \usepackage[linesnumbered, ruled, vlined]{algorithm2e}

%% automata circle and edges related
% \usepackage{tikz}
% \usetikzlibrary{automata,positioning}

% other package
\usepackage{amssymb} % empty set package
\usepackage{url} % url reference
\usepackage{multirow} % multiple rows/columns in table

% insert c code
\usepackage{xcolor}
\usepackage{listings}

\definecolor{mGreen}{rgb}{0,0.6,0}
\definecolor{mGray}{rgb}{0.5,0.5,0.5}
\definecolor{mPurple}{rgb}{0.58,0,0.82}
\definecolor{backgroundColour}{rgb}{0.95,0.95,0.92}

\lstdefinestyle{CStyle}{
    backgroundcolor=\color{backgroundColour},   
    commentstyle=\color{mGreen},
    keywordstyle=\color{magenta},
    numberstyle=\tiny\color{mGray},
    stringstyle=\color{mPurple},
    basicstyle=\footnotesize,
    breakatwhitespace=false,         
    breaklines=true,                 
    captionpos=b,                    
    keepspaces=true,                 
    numbers=left,                    
    numbersep=5pt,                  
    showspaces=false,                
    showstringspaces=false,
    showtabs=false,                  
    tabsize=2,
    language=C
}


% change to a new line when necessary
\makeatletter 
\def\UrlAlphabet{%
      \do\a\do\b\do\c\do\d\do\e\do\f\do\g\do\h\do\i\do\j%
      \do\k\do\l\do\m\do\n\do\o\do\p\do\q\do\r\do\s\do\t%
      \do\u\do\v\do\w\do\x\do\y\do\z\do\A\do\B\do\C\do\D%
      \do\E\do\F\do\G\do\H\do\I\do\J\do\K\do\L\do\M\do\N%
      \do\O\do\P\do\Q\do\R\do\S\do\T\do\U\do\V\do\W\do\X%
      \do\Y\do\Z}
\def\UrlDigits{\do\1\do\2\do\3\do\4\do\5\do\6\do\7\do\8\do\9\do\0}
\g@addto@macro{\UrlBreaks}{\UrlOrds}
\g@addto@macro{\UrlBreaks}{\UrlAlphabet}
\g@addto@macro{\UrlBreaks}{\UrlDigits}
\makeatother

\setlength{\parindent}{0pt} % no indent in the beginning of a paragraph


\title{CSE 451/851 Homework 2}
\author{Tian Gao}
\begin{document}
\maketitle

1.\\
Multiprogramming and time-sharing systems share the following feature in common:
\begin{enumerate}
    \item Both of them keep multiple jobs simultaneously in the memory.
    \item Both of them support multiple users.
\end{enumerate}

1(a).\\
Multiprogramming and time-sharing systems are different in the following features:
\begin{enumerate}
    \item The goal of a multiprogramming system is to maximize CPU use while the goal of timesharing system is to minimize response time for users.
    \item In a multiprogramming system, user interaction is not provided while in a timesharing system, each user can interact with programs and get a quick response.
    \item In a multiprogramming system, the CPU organizes and executes jobs via job scheduling while in time-sharing systems, the CPU executes multiple jobs by switching among them.
\end{enumerate}

1(b).\\
Multi-user system like the CSE server is time-sharing.\\
Justify:\\
We can interact with the CSE server and get a response quickly.
Therefore it's a time-sharing system.

Batch Processing System is multiprogramming.\\
Justify:\\
Batch Processing System organizes code and data into batches and jobs are processed in the order.
The CPU will not switch among jobs.
Therefore it's a multiprogramming system.

2.\\
Multiprogramming is key to a modern operating system’s operation for the following reasons:
\begin{enumerate}
    \item It increases CPU utilization. The CPU will not be idle since it always has one job to execute.
    \item It increases the throughput of the jobs. Since the CPU does not need to wait for the I/O, the number of executed jobs increases, given a certain period of time.
    \item It allows multiple users to visit the system.
\end{enumerate}

3.\\
(a).\\
The number of created unique processes is 6.

(b).\\
Two processes call pthread\_create(), so the number of created unique threads is 2.
However, considered that each process starts as a single thread, the number of total threads should be 8. (2 threads created by pthread\_create() and 6 threads created with the 6 single-threaded processes)

4.\\
The examples of events are listed as follows:\\
Event 1: The process is selected and assigned to the CPU. \\
Event 2: The CPU time of the process has expired.\\
Event 3: The process needs to print a document and is waiting for the printer.\\
Event 4: The process finishes printing the document and is waiting to be executed in memory.\\
Event 5: The process was waiting to be executed in memory but then removed from main memory due to lack of memory space.\\
Event 6: The process was waiting for the printer in memory but then removed from main memory due to lack of memory space.\\

5.\\
The code has been tested on the CSE server.

\begin{lstlisting}[style=CStyle]
    #include <ctype.h>
    #include <stdio.h>
    #include <stdlib.h>
    
    #define MSGSIZE 16
    
    int main()
    {
        char *msg1 = "How are you?";
        char inbuff[MSGSIZE];
        int p1[2], p2[2], child1, child2;
    
        pipe(p1);
        pipe(p2);
        child1 = fork();
    
        if (child1 > 0)
        {
            // parent
        }
        else
        {
            child2 = fork();
            if (child2 > 0)
            {
                // child1
                write(p1[1], msg1, MSGSIZE);
                close(p1[1]);
    
                read(p2[0], inbuff, MSGSIZE);
                close(p2[0]);
                printf("%s\n", inbuff);
            }
            else
            {
                // child2
                read(p1[0], inbuff, MSGSIZE);
                close(p1[0]);
    
                myStrupr(inbuff);
                write(p2[1], inbuff, MSGSIZE);
                close(p2[1]);
            }
        }
        close(p1[0]);
        close(p1[1]);
        close(p2[0]);
        close(p2[1]);
    
        exit(0);
    }
    
    void myStrupr(char *s)
    {
        char *p = s;
        while (*p != (char *)NULL)
        {
            *p = _toupper(*p);
            ++p;
        }
    }
\end{lstlisting}

\end{document}


\documentclass[a4paper]{article}

\usepackage[english]{babel}
\usepackage[utf8]{inputenc}
\usepackage{amsmath}
\usepackage{graphicx}
\usepackage{algorithm}
\usepackage[noend]{algpseudocode}
\usepackage[colorinlistoftodos]{todonotes}
\usepackage{multirow}
\usepackage{parskip}
\usepackage{enumerate}
\usepackage{url}

% empty set package
\usepackage{amssymb}

\title{CSCE 835, Fall 2017, homework 7}
\author{Tian Gao}
\begin{document}
\maketitle


% 1
1.\\
In the code of the pingpong method, "MPI\_Wtime" is used to measure the communication time.\\
The running time for executing "MPI\_Send" 100 times as follows.\\
For intra-node	communication(1 node, 2 tasks for each node), the running time is: 0.000263 seconds.\\
For inter-node	communication(2 nodes, 1 task for each node), the running time is: 0.006898 seconds.\\

% 2
2.\\
To measure the difference between non-blocking and locally blocking send routines, addition operation is executed before the requrest is received.\\
There are 4 cases: \\
If receiving is blocked and sending is not blocked, addition operation is executed for 2083 times.\\
If receiving is not blocked and sending is blocked, addition operation is executed for 4443 times.\\
If neither receiving nor sending is blocked, addition operation is executed for 8486 times.\\
If both receiving and sending are blocked, addition operation is not executed .\\
\end{document}

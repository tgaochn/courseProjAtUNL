\documentclass{beamer}
\usepackage[utf8]{inputenc}
\usepackage[czech]{babel}
\usepackage{array}
\usepackage{amsthm}
\usepackage{graphicx}
\usepackage{listings}
\usepackage{subfig}
\usepackage{float}
\usepackage{multirow}
\usepackage{booktabs}
\usepackage{color}
\usepackage{alltt}
\usepackage{algorithm}
\usepackage{algpseudocode}

% beamer settings
\usetheme{Berlin}
\newtranslation[to=Czech]{Definition}{Definice}

% additional commands
\newcommand{\setHelper}[1]{\left\lbrace #1 \right\rbrace}
\floatname{algorithm}{Algoritmus}
\renewcommand{\algorithmiccomment}[1]{// #1}

% title page
\title[Paralelizace CA]{[07B]- Implementace CA pomocí \\ Python + MPI}
\author[M. Munzar]{Milan Munzar}
\institute[FIT VUT]{Fakulta informačních technologií VUT v Brně}

\begin{document}
%1
\frame{\maketitle}

%2
\section{Zadání}
\begin{frame}
\frametitle{Zadání projektu}
    \begin{itemize}
        \item Implementovat Conwayovu hru Life a alespoń jeden problém ve vícestavovém CA.
        \item Povolené prostředky: \texttt{Python + mpi4py}.
        \item Provést měření nad různými stupni paralelizace.
        \item Porovnat se sekvenční verzí.
        \item Vyhodnotit výsledky.
     \end{itemize}
\end{frame}

%3
\begin{frame}
\frametitle{Conwayova hra Life}
    \begin{itemize}
    \item dvourozměrný uniformní celulární automat
    \item buňka může být ve stavu živá/mrtvá
    \item pravidla hry:
      \begin{itemize}
          \item živá buňka s méně než dvěma živými sousedy zemře
          \item živá buňka s více než třemi živými sousedy zemře
          \item mrtvá buňka se třemi živými sousedy ožije
       \end{itemize}
    \end{itemize}
\end{frame}


\section{Sekveční verze}
%4
\begin{frame}
\frametitle{Hra Life pseudokód}
  \begin{algorithm}[H]
  \caption{sequential Life}
  \begin{algorithmic}[1]
    \State $curr\_epoch \leftarrow 0$
    \While{$curr\_epoch < no\_epoch$}
      \For{$x = 1$ to $matrix\_size - 1$}
        \For{$y = 1$ to $matrix\_size - 1$}
            \State $sum \leftarrow noNeighbor(x, y, matrix1)$
            \State $matrix2[x][y] \leftarrow applyRules(sum, matrix1[x][y])$ 
        \EndFor
      \EndFor
      \State $swap(matrix1, matrix2)$
      \State $curr\_epoch \leftarrow curr\_epoch + 1$
    \EndWhile
  \end{algorithmic}
  \end{algorithm}
\end{frame}


%5
\begin{frame}
\frametitle{Okrajové podmínky}
    \begin{itemize}
          \item Okraji matice jsou napevno přiděleny mrtvé buňky.
          \item Jiná možnost je svázat okraje.
    \end{itemize}
\end{frame}


%6
\begin{frame}
\frametitle{Použitá optimalizace}
   \begin{itemize}
      \item Platí, že pokud buňka ani žádný její soused nezměnil v minulém kroku stav,
pak v tomto kroku buňka svůj stav měnit nebude.
      \item Přidána matice aktivních buňek.
   \end{itemize}
\end{frame}

\section{Paralelní verze}

%7
\begin{frame}
\frametitle{Paralelizace hry Life}
\begin{itemize}
  \item \texttt{MPI\_scatter}
\end{itemize}
\begin{figure}
    \centering
    \includegraphics[scale=0.2]{paralelizace.png}
\end{figure}
\end{frame}


%8
\begin{frame}
\frametitle{Počítání na hranách matic}
\begin{itemize}
  \item Získání součtu z osmiokolí.
  \item Sousedé si vyměńují mezisoučty na hranách.
\end{itemize}
\begin{figure}
    \centering
    \includegraphics[scale=0.3]{hrany.png}
\end{figure}
\end{frame}


%9
\begin{frame}
\frametitle{Paralelní hra Life pseudokód}
  \begin{algorithm}[H]
  \caption{parallel Life}
  \begin{algorithmic}[1]
    \State $curr\_epoch \leftarrow 0$
    \While{$curr\_epoch < no\_epoch$}
      \State $request \leftarrow MPI\_Irecv(borders)$
      \State $borders \leftarrow computeBorders(matrix1)$
      \State $MPI\_Isend(borders)$
      \State $matrix2 \leftarrow computeInnerMatrix(matrix1)$
      \State $MPI\_Wait(request)$
      \State $updateBorders(borders, matrix2)$
      \State $computeCells(matrix1, matrix2)$
      \State $swapPointers(matrix1, matrix2)$
      \State $curr\_epoch \leftarrow curr\_epoch + 1$
    \EndWhile
  \end{algorithmic}
  \end{algorithm}
\end{frame}

\section{Experimenty}

%9
\begin{frame}
\frametitle{Experimenty}
Testováno na stroji s \texttt{Intel Core(TM) i5-3210M CPU @ 2.50GHz} a 8GB operační paměti.
\end{frame}

\begin{frame}
\frametitle{Doba výpočtu při různých stupních paralelizace}
\begin{figure}
    \centering
    \includegraphics[scale=0.55]{doba.png}
\end{figure}
\end{frame}

\begin{frame}
\frametitle{Dosažené zrychlení}
\begin{figure}
    \centering
    \includegraphics[scale=0.6]{zrychleni.png}
\end{figure}
\end{frame}

\section{Závěr}

\begin{frame}
\frametitle{Použité prostředky}
\begin{itemize}
  \item \texttt{Python 2.7}
  \begin{itemize}
    \item \texttt{Tkinter, threading, Queue}
    \item \texttt{numpy}
    \item \texttt{mpi4py}
    \item \texttt{timeit}    
  \end{itemize}
\end{itemize}
\end{frame}

\end{document}

                           
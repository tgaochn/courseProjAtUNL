\documentclass[12pt]{article}
\usepackage[utf8]{inputenc}
\usepackage[czech]{babel}
\usepackage[a4paper, top=2cm, bottom=2cm, left=2cm, right=2cm]{geometry}
\usepackage[IL2]{fontenc}
\usepackage{url}
\usepackage{indentfirst}
\usepackage{amsthm}
\usepackage{amsmath}
\usepackage{amsfonts}
\usepackage{graphicx}
\usepackage{caption}
\usepackage{subcaption}
\usepackage{color}
\usepackage{float}
\usepackage{multirow}
\usepackage{booktabs}
\usepackage[table,usenames,dvipsnames,svgnames]{xcolor}
\usepackage[unicode,hyperindex,plainpages=false,pdftex]{hyperref}
\usepackage{algorithm}
\usepackage[noend]{algpseudocode}   
    \hypersetup{
          colorlinks=true, 
          linkcolor=Red, 
          citecolor=Red, 
          filecolor=magenta, 
          urlcolor=TealBlue
    }

\newcommand{\todo}[1]{\textcolor{red}{[[TODO: #1]]}}
\newcommand{\setHelper}[1]{\left\lbrace #1 \right\rbrace}
\newlength{\skipeqarray}
\setlength{\skipeqarray}{0.4cm}


\title{[07B]- Implementace CA pomocí Python + MPI}
\author{Milan Munzar \\
\normalsize{\href{mailto:xmunza00@stud.fit.vutbr.cz}{xmunza00@stud.fit.vutbr.cz}}}
\date{}

\definecolor{lightgray}{rgb}{0.9, 0.9, 0.9}
\newtheorem{veta}{Věta}
\newtheorem{definition}{Definice}
\newtheorem{priklad}{Příklad}
\newfloat{algorithm}{h}{lop}
\floatname{algorithm}{Algoritmus}
\renewcommand{\algorithmiccomment}[1]{\hfill\# #1}
\algrenewcommand\alglinenumber[1]{{\footnotesize#1}}
    
\begin{document}
\maketitle

\section{Úvod}
Tento text popisuje paralelní řešení celulárního automatu realizující hru Life. Program je implementován v jazyce Python s využítím knihovny \texttt{mpi4py}. V sekci~\ref{porov} se nachází porovnání doby běhu programu s různým stupněm paralelizace a sekvenční verze.

\section{Conwayova hra Life}
Conwayova hra Life je uniformní dvourozměrný celulární automat. Mřížka hry obsahuje buňky, které mohou být ve stavu živá/mrtvá. Stav buňky závisí na jejím osmiokolím. Pravidla hry jsou následujicí: živá buňka s méně než dvěma živými sousedy zemře, živá buňka s více než třemi živými sousedy zemře a mrtvá buňka se třemi živými sousedy ožije

Výpočet mřížky jsem urychlil sledováním aktivity buněk. Pokud buňka a její sousedé zůstali v minulém kroku výpočtu mřížky nezměněni, pak ani v tomto kroku se stav buňky nemění. Jako okrajovou podmínku jsem použil pevné přiřazení stavu mrtvá krajním buňkám.

\section{Paralelizace Conwayovy hry Life}
\label{implemLife}
Výpočet mřížky mezi procesory jsem rozdělil způsobem zobrazeným na obrázku~\ref{div}. Každý proces dostane $radku\_matice/pocet\_procesoru$ řádků a nad nimi poté provádí výpočet. Tato operace je zajištěna funkcí \texttt{scatter()}. Sběr výsledků se prováde funkcí \texttt{gather()}.

\begin{figure}
    \centering
    \includegraphics[scale=0.2]{paralelizace.png}
    \caption{Paralelizace výpočtu mřížky}
    \label{div}
\end{figure}

Aby byl výsledek výpočtu validní musí si procesory vyměňovat informace o počtu živých buňěk na sousedicích hranách. K tomu jsem použil dodatečné pole reprezentující mezisoučet na hranách matice. Toto pole si procesy vymění pomocí neblokujícíh operací \texttt{Isend()} a \texttt{Irecv()}. Použití neblokující komunikace mi dovoluje překrýt výpočet vnitřní matice s výměnou hraničních polí.
Po výpočtu vnitřní matice se dokončí výpočet hran (funkcí \texttt{Wait()} čekám na dokončení komunikace) a uzavře se jedna iterace algoritmu.


\section{Zhodnocení paralelní verzí programu}
\label{porov}

Měření byla prováděna na stroji s dvoujádrovým procesorem \texttt{Intel Core(TM) i5-3210M CPU @ 2.50GHz} a 8GB operační paměti. Porovnání doby běhu programu paralelních a sekvenční verzí je vidět na obrázku~\ref{doba}. Dosažená zrychlení jsou vidět na obr.~\ref{zrychleni}. Hodnoty vynesené v grafu jsou brány jako průměrná hodnota ze 3 běhů programu nad aktuálním nastavení mřížky. Počet epoch byl ve všech měřeních 100.

\begin{figure}
    \centering
    \includegraphics[scale=0.6]{doba.png}
    \caption{Doba běhu programu při různých stupních paralelizace}
    \label{doba}
\end{figure}

\begin{figure}
    \centering
    \includegraphics[scale=0.6]{zrychleni.png}
    \caption{Dosažená zrychlení}
    \label{zrychleni}
\end{figure}

\section{Závěr}

Podařilo se mi naimplementovat sekvenční i paralelní variantu Conwayovy hry Life. K dodaným skriptům jsem vytvořil jenoduché GUI. Bohužel dosažená zrychlení nejsou veliká. Chybou může být neefektivní implementace a nevhodnost použitého stroje pro měření paralelních programů. Zadání jsem splnil pouze částečně, protože jsem nestihl implementovat alespoň jeden problém vícestavovém prostoru.

\newpage
\section*{Appendix}
\begin{itemize}
\item Příklad spuštění paralelního programu se třemi procesy:
\begin{verbatim}
mpiexec -n 3 python prl_life.py 
\end{verbatim}
\end{itemize}
\end{document}

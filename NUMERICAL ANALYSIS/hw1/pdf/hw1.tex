\documentclass[a4paper]{article}

\usepackage[english]{babel}
\usepackage[utf8]{inputenc}
\usepackage{amsmath}
\usepackage{graphicx}
\usepackage{algorithm}
\usepackage[noend]{algpseudocode}
\usepackage[colorinlistoftodos]{todonotes}
\usepackage{multirow}
\usepackage{parskip}
\usepackage{enumerate}
\usepackage{url}

% change to a new line when necessary
\makeatletter
\def\UrlAlphabet{%
      \do\a\do\b\do\c\do\d\do\e\do\f\do\g\do\h\do\i\do\j%
      \do\k\do\l\do\m\do\n\do\o\do\p\do\q\do\r\do\s\do\t%
      \do\u\do\v\do\w\do\x\do\y\do\z\do\A\do\B\do\C\do\D%
      \do\E\do\F\do\G\do\H\do\I\do\J\do\K\do\L\do\M\do\N%
      \do\O\do\P\do\Q\do\R\do\S\do\T\do\U\do\V\do\W\do\X%
      \do\Y\do\Z}
\def\UrlDigits{\do\1\do\2\do\3\do\4\do\5\do\6\do\7\do\8\do\9\do\0}
\g@addto@macro{\UrlBreaks}{\UrlOrds}
\g@addto@macro{\UrlBreaks}{\UrlAlphabet}
\g@addto@macro{\UrlBreaks}{\UrlDigits}
\makeatother

% empty set package
\usepackage{amssymb}

\title{CSCE 440/840, Spring 2018, Homework 1}
\author{Tian Gao}
\begin{document}
\maketitle


% 1
1.\\
(a).\\
$\because 2^{-14} = (-1)^0 \times 2^{113 - 127} \times (1 + 0)$\\
$\therefore s = 0, E = 113, M = 0$\\
$\therefore$ The answer is 0 1110001 00000000000000000000000\\

(b).\\
$\because -1.5 = (-1)^1 \times 2^{127 - 127} \times (1 + 2^{-1})$\\
$\therefore s = 1, E = 127, M = 2^{-1}$\\
$\therefore$ The answer is 1 1111111 10000000000000000000000\\

(c).\\
$\because 32.015725 \approx (-1)^0 \times 2^{132 - 127} \times (1 + 2^{-11} + 2^{-19} + 2^{-20} + 2^{-22})$\\
$\therefore s = 0, E = 132, M = 2^{-11} + 2^{-19} + 2^{-20} + 2^{-22}$\\
$\therefore$ The answer is 0 0000100 00000000001000000011010\\

(d).\\
$\because -10^{-9} \approx (-1)^1 \times 2^{97 - 127} \times (1 + 2^{-4} + 2^{-7} + 2^{-9} + 2^{-10} + 2^{-11} + 2^{-17} + 2^{-19} + 2^{-20} + 2^{-21} + 2^{-22} + 2^{-23})$\\
$\therefore s = 0, E = 113, M = 2^{-4} + 2^{-7} + 2^{-9} + 2^{-10} + 2^{-11} + 2^{-17} + 2^{-19} + 2^{-20} + 2^{-21} + 2^{-22} + 2^{-23}$\\
$\therefore$ The answer is 0 1100001 00010010111000001011111\\

% 2
2.\\
(a).\\
$\because (10000101)_2 = (133)_{10}$\\
$\therefore s = 0, E = 133, M = 2^{-1} + 2^{-2} + 2^{-3} + 2^{-4} + 2^{-5} + 2^{-6}$\\
Using formula $v = (-1)^0 \times 2^{E - 127} \times (1 + M) $\\
The answer is 127.0\\

(b).\\
$\because (10000010)_2 = (130)_{10}$\\
$\therefore s = 1, E = 130, M = 2^{-2}$\\
Using formula $v = (-1)^1 \times 2^{E - 127} \times (1 + M) $\\
The answer is -10.0\\

(c).\\
$\because (10000010)_2 = (130)_{10}$\\
$\therefore s = 0, E = 130, M = 2^{-1} + 2^{-2} + 2^{-3} + 2^{-4} + 2^{-10} + 2^{-11} + 2^{-12} + 2^{-13} + 2^{-14} + 2^{-15} + 2^{-16} +  2^{-17} + 2^{-18} + 2^{-21} + 2^{-22}$\\
Using formula $v = (-1)^0 \times 2^{E - 127} \times (1 + M) $\\
The answer is 15.5156\\

(d).\\
$\because (01011101)_2 = (93)_{10}$\\
$\therefore s = 0, E = 93, M = 2^{-1} + 2^{-2} + 2^{-3} + 2^{-5} + 2^{-6} + 2^{-7} + 2^{-9} + 2^{-10} + 2^{-14} + 2^{-15} + 2^{-16} +  2^{-17} + 2^{-19} + 2^{-20} + 2^{-21} + 2^{-22} + 2^{-23}$\\
Using formula $v = (-1)^0 \times 2^{E - 127} \times (1 + M) $\\
The answer is $1.125 \times 10^{-10}$\\

(e).\\
$\because (11111111)_2 = (255)_{10}$\\
$\therefore s = 0, E = 255, M = 0$\\
Using formula $v = (-1)^0 \times 2^{E - 127} \times (1 + M) $\\
The answer is $2^{128} = 340282366920938463463374607431768211456$\\

(f).\\
$\because (10110110)_2 = (182)_{10}$\\
$\therefore s = 0, E = 182, M = 2^{-4} + 2^{-9} + 2^{-14} + 2^{-18} + 2^{-19}$\\
Using formula $v = (-1)^1 \times 2^{E - 127} \times (1 + M) $\\
The answer is $3.8353371 \times 10^{16}$\\

% 3
3.\\
$\because f(x)=f(x_0) + \sum\limits_{t=1}^{\infty}\frac{f^{(t)}(x_0)}{t!}(x-x_0)^t$\\
$\therefore f(3 + h) = f(3) + f'(3)h + \frac{f''(3)}{2!}h^2 + ... + \frac{f^{(n)}(3)}{n!}h^n$\\
$\because f(3) = -71, f'(3) = -72, f''(3)=-42, f'''(3)=-12, f''''(3) = 0$\\
$\therefore f(3 + h) = 2h^3 -21h^2-72h-71$\\
By bringing x = 3 + h directly into f(x), we can also get
$f(3 + h) = 2h^3 -21h^2-72h-71$\\
So the result is verified.\\

% 4
4.\\
The code wrote in MATLAB can be found in findingZero.m and run.m.\\
The code has been tested in MATLAB R2017a.\\
To run the programe, just put the two scripts in the same folder and run run.m.\\
Using Bisection Method, the root is 0.4597.\\
Using Newton's Method, the root is 0.4597.\\
Using Secant Method, the root is 0.4597.\\

% 5
5.\\
The fixed point of g(x) is the root of equation g(x) = x.\\
Let f(x) = g(x) - x, then the root of f(x) = 0 is the root of the above equation.\\
Using the same programe of Newton's Method in question 4, we can get the fixed point of g(x) is 0.9032.\\
\end{document}
